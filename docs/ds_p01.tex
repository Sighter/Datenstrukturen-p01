
% documentation for for first practice of cg1

\documentclass{article}
\usepackage{times}
\usepackage{german}
\usepackage{fancyhdr}
\usepackage[utf8]{inputenc}

\title{Datenstrukturen\\Praxiseinheit 1}
\author
{
	Sascha Ebert\\
	MatNr: 177182\\
	\texttt{sascha.ebert@s2006.tu-chemnitz.de}\\
	\texttt{https://github.com/Sighter/Datenstrukturen-p01}\\
}

\date{\today}

\begin{document}
\maketitle

\begin{abstract}
Anbei finden sie die Lösungen zu den nicht programmierbaren Aufgaben der 1. Praxiseinheit.
Alle anderen Programmcode-bezogenen Aufgaben finden sie im Quelltext an den jeweiligen Positionen
erläutert. Es somit auch möglich das oben vermerkte Git-Repository zu nutzen um den Quelltext
zu betrachten. Die aufgabenbezogenen Kommentare sind mit `Exercise' oder `Hint' gekenntzeichnet.
\end{abstract}

\section*{Aufgabe 3}
\paragraph{Frage:}
Schreiben Sie eine Funktion, um nach einem Freund über dessen Nummer zu suchen!
Geht das schneller als in O(n) (wobei n die Zahl der Einträge in der Freundesliste ist)?

\paragraph{Antwort:}
Es ist möglich die Suche schneller als in \(O(n)\) zu schreiben, da das zugrunde liegende Array
geordnet ist. Als Beispiel wäre die binäre suche zu nennen, welche maximal \(\log_{2}(n)\) Schritte
benötigt.

Eine weitere Möglichkeit wäre die Nummern der Freunde als Feldindizes zu benutzen. Dabei wäre der Zugriff
mit konstanter Komplexität möglich. Dies ist aber durch die Freiwählbarkeit der Nummern durch den Nutzer
nahezu ausgeschlossen.

\section*{Aufgabenzuordnung}
\begin{tabular}{ l l }
  Aufgabe 1  & friendlist.h \\
  Aufgabe 2  & friendlist.cpp \\
  Aufgabe 4  & friendlist.cpp \\
  Aufgabe 5  & interface.cpp \\
  Hinweis 2  & readline\_dyn.cpp
\end{tabular}

\end{document}


